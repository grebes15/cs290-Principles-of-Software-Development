\documentclass{article}
\begin{document}
\begin{center}
\textbf{Computer Science 290 Team 3 Final Project}
Requirements Document
\end{center}
\noindent
\textbf{\Large Design Specification}
\\
For the final project, the TwitterAnalytics system implements five analysis tools on Twitter tweets by a user. The system must be able to perform these analyses on a .zip archive provided by the user, and also on their live timeline. It should store all of these tweets in a SQLite 3 database.. The tool will be required to authenticate with the Twitter REST API version 1.1 to connect a user with their most recent tweets. In addition, the system must implement a clearing functionality to reset the internal database, and must also allow the use of multiple zip file inputs.
Additionally the system must implement a command line interface and parse these inputs using the JCommander software tool.
\\
\\
\noindent
\textbf{\Large System Behavior}
\\
Our system will implement a graphical user interface (GUI) and a command line interface for our users to interact with. The user interfaces will expose all system functionality as specified above, giving the user a large amount of control over their data and analytics.
\\
\\
\\
\\
\textit{\Large 1. Command Line Interface}
\\
\\
\\
Usage:
\\
java -jar burdwatch.jar [OPTION...] [ANALYZER]
\\
Performs analysis on the available tweet data using the specified analyzer
\\
\\
\parbox[t][3cm][t]{7cm}{\normalsize date\\
\\
\\
pattern PATTERN\\
\\
\\
ratio
\\
\\
\\
\\
device
\\
\\
\\
most-retweeted} 
\parbox[t][3cm][t]{7cm}{\normalsize Identifes the date on which the most tweets occurred\\
\\
Prints all tweets that contain the given pattern\\
\\
Prints the user most retweeted. Requires an Internet connection to retrieve the username, otherwise prints the ID number.\\ 
\\
Prints the ratio of retweets to non-retweet tweets\\
\\
Identifies and prints the device most used to post tweet\\}
\\
\\
\\
\\
\\
\\
\\
\\
\\
\\
\\
\\
\\
\\
\noindent
OPTIONS:

\parbox[t][3cm][t]{7cm}{\normalsize -auth FILE\\
\\
\\
\\
-oa FILE\\
\\
\\
-zip FILE\\
\\
\\
--verbose -v
\\
\\
\\
--clear -c
} 
\parbox[t][3cm][t]{7cm}{\normalsize Connect to the Internet by performing \\Twitter authentication, and write token result to t he specified file for later\\
\\
Connect to the Internet using the OAuth file created previously by the -auth option\\
\\
Adds the contents of the specified \\Twitter archive file to the internal database.\\
\\
Verbose mode, which prints out \\technical errors messages\\
\\
Clears the database first before doing anything else}
\\
\\
\\
\\
\\
\\
\\
\\
\\
\\
\\
\\
\\
\\
\\
\\
\\
\\
\textit{\Large 2. Analytics}
\\
The following twitter analytics will be supported by the system:
\begin{itemize}
\item All tweets that contain a specified character pattern
\item The localized string representing the day of the week on which this user tweeted the most, formatted to the user's locale
\item The user most commonly retweeted by this user
\item The ratio of retweets to non-retweets
\item The device most used by this user to send tweets
\end{itemize} 











\end{document}

